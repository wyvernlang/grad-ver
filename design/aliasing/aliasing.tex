\section{Aliasing}

\subsection{Definitions}

An \tbf{object variable} is one of the following:
\begin{itemize}
  \item a class instance variable i.e. a variable $v$ such that $v : C$ for some class $C$.
  \item a class instance field reference i.e. a field reference $e.f$ where $e.f : C$ for some class $C$.
\end{itemize}
Let $\O$ be a set of object variables.
An $O \subset \O$ \tbf{aliases} if and only if
each $o \in O$ refers to the same memory in the heap as each other,
written propositionally as
$$
  \forall o, o\p \in O : o = o\p \iff \aliases(O)
$$
An $O \subset \O$ \tbf{non-aliases} if and only if
each $o \in O$ refers to separate memory in the heap as each other,
written propositionally as
$$
  \forall o, \tilde o : o \neq \tilde o \iff \nonaliases(O)
$$

\subsection{Aliasing Context}

Let $\phi$ be a formula.
The \tbf{aliasing context} $\A$ of $\phi$ is a tree of set of aliasing proposition about aliasing of object variables that appear in $\phi$.
$\A$ needs to be a tree because the conditional sub-formulas that may appear in $\phi$ allow for branching aliasing contexts not expressible flatly at the top level.
Each node in the tree corresponds to a set of aliasing propositions, and each branch refers to a branch of a unique conditional in $\phi$.
The parts of the tree are labeled in such a way that modularly allows a specified sub-formula of $\phi$ to be matched to the unique aliasing sub-context that corresponds to it.
\\
For example, consider the following formula:
$$
  \phi :=
  (o_1 = o_2) \ \ast \
  (\cif \ (b_1)
    \ \cthen \ (
      (o_1 \neq o_3) \ \ast \
      (\cif \ (b_2)
        \ \cthen \ (o_1 = o_4) \
        \ \celse \ (b_3)
      )
    )
    \ \celse \ (o_1 = o_3)
  ) \ \ast \
  (o_1 = o_4)
$$
$\phi$ has a formula-structure represented by the following tree:
\\
%
\input{aliasing/aliasing-context-formula-example.tex}
%
The formula-structure tree for $\phi$ corresponds node-for-node and edge-for-edge to the following aliasing context:
\\
%
% Start of code
% \begin{tikzpicture}[anchor=mid,>=latex',line join=bevel,]
\begin{center}
\begin{tikzpicture}[>=latex',line join=bevel,]
  \pgfsetlinewidth{1bp}
%%
\pgfsetcolor{black}
  % Edge: C1 -> C2
  \draw [->] (142.54bp,183.82bp) .. controls (134.04bp,170.34bp) and (122.32bp,151.75bp)  .. (107.44bp,128.14bp);
  \definecolor{strokecol}{rgb}{0.0,0.0,0.0};
  \pgfsetstrokecolor{strokecol}
  \draw (160.0bp,156.0bp) node[right=-3.5cm] {$\texttt{if} \ b_1 \ \texttt{then}$};
  % Edge: C1 -> C5
  \draw [->] (172.42bp,183.58bp) .. controls (181.28bp,174.72bp) and (190.0bp,166.0bp)  .. (190.0bp,166.0bp) .. controls (190.0bp,166.0bp) and (195.6bp,151.74bp)  .. (204.92bp,128.03bp);
  \draw (214.5bp,156.0bp) node[right=-2cm] {$\texttt{else}$};
  % Edge: C2 -> C3
  \draw [->] (80.648bp,91.578bp) .. controls (73.267bp,82.721bp) and (66.0bp,74.0bp)  .. (66.0bp,74.0bp) .. controls (66.0bp,74.0bp) and (60.907bp,59.739bp)  .. (52.438bp,36.027bp);
  \draw (95.0bp,64.0bp) node[right=-3.5cm] {$\texttt{if} \ b_2 \ \texttt{then}$};
  % Edge: C2 -> C4
  \draw [->] (110.33bp,91.578bp) .. controls (117.22bp,82.721bp) and (124.0bp,74.0bp)  .. (124.0bp,74.0bp) .. controls (124.0bp,74.0bp) and (129.6bp,59.739bp)  .. (138.92bp,36.027bp);
  \draw (148.5bp,64.0bp) node[right=-2cm] {$\texttt{else}$};
  % Node: C1
\begin{scope}
  \definecolor{strokecol}{rgb}{0.0,0.0,0.0};
  \pgfsetstrokecolor{strokecol}
  % \draw (218.0bp,220.0bp) -- (90.0bp,220.0bp) -- (90.0bp,184.0bp) -- (218.0bp,184.0bp) -- cycle;
  \draw (210.0bp,220.0bp) -- (100.0bp,220.0bp) -- (100.0bp,184.0bp) -- (210.0bp,184.0bp) -- cycle;
  \draw (154.0bp,202.0bp) node {$\aliased\set{o_1, o_2, o_4}$};
\end{scope}
  % Node: C2
\begin{scope}
  \definecolor{strokecol}{rgb}{0.0,0.0,0.0};
  \pgfsetstrokecolor{strokecol}
  \draw (147.5bp,128.0bp) -- (44.5bp,128.0bp) -- (44.5bp,92.0bp) -- (147.5bp,92.0bp) -- cycle;
  \draw (96.0bp,110.0bp) node {$\empty$};
\end{scope}
  % Node: C5
\begin{scope}
  \definecolor{strokecol}{rgb}{0.0,0.0,0.0};
  \pgfsetstrokecolor{strokecol}
  \draw (258.0bp,128.0bp) -- (166.0bp,128.0bp) -- (166.0bp,92.0bp) -- (258.0bp,92.0bp) -- cycle;
  \draw (212.0bp,110.0bp) node {$\aliased\set{ o_1, o_3 }$};
\end{scope}
  % Node: C3
\begin{scope}
  \definecolor{strokecol}{rgb}{0.0,0.0,0.0};
  \pgfsetstrokecolor{strokecol}
  \draw (92.0bp,36.0bp) -- (0.0bp,36.0bp) -- (0.0bp,0.0bp) -- (92.0bp,0.0bp) -- cycle;
  \draw (46.0bp,18.0bp) node {$\aliased\set{ o_1, o_4 }$};
\end{scope}
  % Node: C4
\begin{scope}
  \definecolor{strokecol}{rgb}{0.0,0.0,0.0};
  \pgfsetstrokecolor{strokecol}
  \draw (182.0bp,36.0bp) -- (110.0bp,36.0bp) -- (110.0bp,0.0bp) -- (182.0bp,0.0bp) -- cycle;
  \draw (146.0bp,18.0bp) node {$\varnothing$};
\end{scope}
%
\end{tikzpicture}
\end{center}
% End of code

%
where a node inherits all the aliasing assertions of its parents.
So for example, the aliasing context for the sub-formula $(o_1 = o_4)$ of $\phi$ is:
$$
  \A_\phi(o_1 = o_4) :=
  \set{
    \aliased\set{o_1, o_2, o_4},
    \pnot \aliased\set{o_1, o_3},
    \pnot \aliased\set{o_2, o_3},
    \pnot \aliased\set{o_3, o_4}
  }
$$
More generally, for $\phi$ a formula and $\phi\p$ a sub-formula of $\phi$, write $\A_\phi(\phi\p)$ as the total aliasing context of $\phi\p$, including inheritance from its place in the aliasing context of $\phi$. Usually $\A_\phi(\phi\p)$ is abbreviated to $\A(\phi\p)$ when the top level formula $\phi$ is implicit.
